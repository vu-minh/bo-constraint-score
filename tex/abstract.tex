\begin{abstract}
In data analysis, visualization through dimensionality reduction is one of the most effective ways to understand a dataset. However, the hyperparameters of those visualization algorithms are sometimes difficult to tune for end-users. Indeed, the problem is how to make these powerful techniques more accessible to end-users who do not have advanced knowledge in machine learning. We propose a solution to ease the choice of hyperparameter values for obtaining good quality visualizations. Users can define their requirements for the expected visualization in term of visual constraints that can be easily collected via an interactive interface. Labels can also be used if this kind of knowledge is provided. Our method then selects the set of visualizations that best fit user, or label, requirements. Our resulting visualizations are compared with several quality metrics to assure the reliability of the proposed method. Without requiring expert knowledge about the datasets nor about complex visualization techniques, a few predefined constraints reflecting the natural cognitive judgments of users are enough to find an adequate visualization.
\end{abstract}

%% \begin{graphicalabstract}
%% \includegraphics{figs/grabs.pdf}
%% \end{graphicalabstract}

%% \begin{highlights}
%% \item Research highlights item 1
%% \item Research highlights item 2
%% \item Research highlights item 3
%% \end{highlights}

\begin{keyword}
%  Machine Learning \sep
Dimensionality Reduction\sep
Visualization \sep
Pairwise Constraints \sep
Hyperparameter Optimization \sep
Bayesian Optimization 
\end{keyword}
